\documentclass[a4paper]{exam}

\usepackage{amsmath,amsthm}
\usepackage{array}
\usepackage[a4paper]{geometry}

\header{CS/MATH 113}{PSet 03: Propositional Logic and Equivalences}{Spring 2024}
\footer{}{Page \thepage\ of \numpages}{}
\runningheadrule
\runningfootrule

\newcolumntype{C}{>{$}c<{$}} % math-mode version of "c" column type
\newcommand\T{\ensuremath{\mathrm{T}}}
\newcommand\F{\ensuremath{\mathrm{F}}}

\title{Problem Set 03: Propositional Logic and Equivalences}
\author{CS/MATH 113 Discrete Mathematics}
\date{Spring 2024}

\boxedpoints

\usepackage{draftwatermark}
\SetWatermarkText{Sample Solution}
\SetWatermarkScale{3}
\printanswers

\begin{document}
\maketitle

\begin{questions}

  \question Show that the following are logically equivalent without using truth tables:
  \begin{parts}
    \part $(p \implies r) \lor (q \implies r) \equiv (p \land q) \implies r$

    \begin{solution}
      We solve LHS to obtain the same expression as RHS.
      \begin{proof}
      \begin{align*}
        (p \implies r) \lor (q \implies r) &\equiv \lnot p \lor r \lor \lnot q \lor r && \text{conditional-disjunction}\\
                                           &\equiv \lnot p \lor \lnot q \lor r \lor r && \text{commutativity}\\
                                           &\equiv \lnot (p \land q) \lor r && \text{De Morgan's, idempotent}\\
                                           &\equiv (p \land q) \implies r && \text{conditional-disjunction}
      \end{align*}
    \end{proof}
  \end{solution}
    \part 
    $\lnot \left[ \lnot \left[\left(p \lor q \right) \land r \right] \lor \lnot q \right] \equiv q \land r $

    \begin{solution}
      We solve LHS to obtain the same expression as RHS.
      \begin{proof}
      \begin{align*}
        \lnot [ \lnot [(p \lor q ) \land r ] \lor \lnot q ] &\equiv [[(p \lor q ) \land r ] \land q ] && \text{De Morgan's}\\
                                                         &\equiv (p \lor q ) \land q  \land r && \text{associativity, commutativity}\\
                                                         &\equiv q \land r && \text{absorption}
      \end{align*}
    \end{proof}
  \end{solution}
    \part  $(p \lor q \lor r) \land (p \lor t \lor \lnot q) \land (p \lor \lnot t \lor r)  \equiv p \lor [ r \land (t \lor \lnot q )]$

    \begin{solution}
      We notice $t\lor \lnot q$ on both sides.\\
      For simplicity, let $X\equiv t\lor \lnot q$.\\
      Then the given identity becomes: $(p \lor q \lor r) \land (p \lor X) \land (p \lor \lnot t \lor r)  \equiv p \lor (r \land X)$.\\
      We solve LHS to obtain the same expression as RHS.
      \begin{proof}
      \begin{align*}
        \text{LHS} &: (p \lor q \lor r) \land (p \lor X) \land (p \lor \lnot t \lor r) & & \\
                   & \equiv (p \lor X) \land (p \lor q \lor r) \land (p \lor \lnot t \lor r)  && \text{commutativity}\\
                   & \equiv (p \lor X) \land ((p \lor r)\lor (q\land \lnot t))  && \text{distribution}\\
                   & \equiv (p \lor X) \land ((p \lor r)\lor \lnot X)  && \text{De Morgan's}\\
                   & \equiv (p\land(p\lor r)) \lor (p\land\lnot X) \lor (X\land(p \lor r)) \lor (X\land\lnot X)  && \text{distribution}\\
                   & \equiv p \lor (p\land\lnot X) \lor (X\land(p \lor r)) \lor \F  && \text{absorption, negation}\\                      & \equiv p \lor (X\land(p \lor r))  && \text{absorption, identity}\\        
                   & \equiv p \lor (p\land X) \lor (r\land X)  && \text{distribution}\\        
                   & \equiv p \lor (r\land X)  && \text{absorption}
      \end{align*}
    \end{proof}
  \end{solution}
  \end{parts}
  
  \question Use Truth tables to see if the following statements are logically equivalent: 
  \begin{parts}
    \part $p \implies (q \lor r) \equiv (q \implies p) \land (p \implies r)$

    \begin{solution}
      \null\newline
      \begin{tabular}{CCC||*{5}{C}}
        p & q & r & q\lor r & q\implies p & p\implies r & p \implies (q \lor r) & (q \implies p) \land (p \implies r)\\\hline\hline
        \T & \T & \T & \T & \T & \T & \T & \T \\
        \T & \T & \F & \T & \T & \F & \T & \F \\
        \T & \F & \T & \T & \T & \T & \T & \T \\
        \T & \F & \F & \F & \T & \F & \F & \F \\
        \F & \T & \T & \T & \F & \T & \T & \F \\
        \F & \T & \F & \T & \F & \T & \T & \F \\
        \F & \F & \T & \T & \T & \T & \T & \T \\
        \F & \F & \F & \F & \T & \T & \T & \T \\
      \end{tabular}
      
      The last two columns corresponding to the two statements are nonidentical. Therefore, the statements are not logically equivalent.
    \end{solution}
    \part 
    $(p \lor q) \implies r \equiv (p \implies r) \land (q \implies r)$

    \begin{solution}
      \null\newline
      \begin{tabular}{CCC||*{5}{C}}
        p & q & r & p\lor q & p\implies r & q\implies r & (p\lor q)\implies r & (p \implies r) \land (q \implies r)\\\hline\hline
        \T & \T & \T & \T & \T & \T & \T & \T \\
        \T & \T & \F & \T & \F & \F & \F & \F \\
        \T & \F & \T & \T & \T & \T & \T & \T \\
        \T & \F & \F & \T & \F & \T & \F & \F \\
        \F & \T & \T & \T & \T & \T & \T & \T \\
        \F & \T & \F & \T & \T & \F & \F & \F \\
        \F & \F & \T & \F & \T & \T & \T & \T \\
        \F & \F & \F & \F & \T & \T & \T & \T \\
      \end{tabular}
      
      The last two columns corresponding to the two statements are identical. Therefore, the statements are logically equivalent.
    \end{solution}
    \part  $p \implies (q \lor r) \equiv \lnot r \implies (p \implies q)$

    \begin{solution}
      The LHS is the same as part (a) so we can copy the column from that table.

      \begin{tabular}{CCC||*{4}{C}}
        p & q & r & \lnot r & p\implies q & p \implies (q \lor r) & \lnot r \implies (p \implies q)\\\hline\hline
        \T & \T & \T & \F & \T & \T & \T \\
        \T & \T & \F & \T & \T & \T & \T \\
        \T & \F & \T & \F & \F & \T & \T \\
        \T & \F & \F & \T & \F & \F & \F \\
        \F & \T & \T & \F & \T & \T & \T \\
        \F & \T & \F & \T & \T & \T & \T \\
        \F & \F & \T & \F & \T & \T & \T \\
        \F & \F & \F & \T & \T & \T & \T \\
      \end{tabular}
      
      The last two columns corresponding to the two statements are identical. Therefore, the statements are logically equivalent.
    \end{solution}
  \end{parts}

  \question Express the negation of each of the following statements in natural language using De Morgan’s laws.
  \begin{parts}
  \part Graduates take a job in industry or go to graduate school or start their own ventures.
  \begin{solution}
    Graduates neither take a job in industry nor go to graduate school nor start their own ventures.
  \end{solution}
  \part First year students know python and calculus.
  \begin{solution}
    First year students do not know at least one of python or calculus.
  \end{solution}
  \part Horizon is new and bright.
  \begin{solution}
    Horizon is old or dull.
  \end{solution}
\end{parts}

\question Determine whether each of the following compound propositions is satisfiable.
\begin{parts}
  \part $(p\lor \lnot q)\land(\lnot p\lor q)\land(\lnot p\lor\lnot q)$
  \begin{solution}
    The proposition is a conjunction of three simpler propositions, so is satisfied when each of the three is True. Each of the simpler propositions is a disjunction so is True when any operand is True. All operands involve $p$, $q$, or their negations. In fact, we see that $(p,q) = (\F,\F)$ makes all three True and thus satisfies the compound proposition.

    Therefore, the compound proposition is satisfiable.
  \end{solution}
  \part $(p\implies q)\land(p\implies\lnot q)\land (\lnot p\implies q)\land(\lnot p\implies\lnot q)$
  \begin{solution}
    The proposition is a conjunction of four simpler propositions, so is satisfied when each of the four is True. Each of the simpler propositions is an implication. All operands involve $p$, $q$, or their negations. In fact, we see that the four propositions cover all possible combinations of truth values of $p$ and $q$. Thus, one of the four propositions will always be False.

    Therefore, the compound proposition is not satisfiable.
  \end{solution}
  \part $(p \iff q)\land(\lnot p \iff q)$
  \begin{solution}
    The proposition is a conjunction of two simpler propositions, so is satisfied when each of the two is True. Each of the simpler propositions is a biconditional. The biconditional statements are such that they will never be True at the same time.

    Therefore, the compound proposition is not satisfiable.
  \end{solution}
\end{parts}
\end{questions}
\end{document}
%%% Local Variables:
%%% mode: latex
%%% TeX-master: t
%%% End: